\documentclass{article}
\usepackage{ctex}
\usepackage{graphicx}
\usepackage{amsmath}
\usepackage{indentfirst}
\usepackage{titlesec}
\usepackage{setspace}
\usepackage{subfigure}
\usepackage{caption}
\usepackage{float}
\usepackage{booktabs}
\usepackage{geometry}
\usepackage{multirow}
\geometry{left=1.2cm,right=1.2cm,top=2cm,bottom=2cm}
\title{\songti \zihao{2}\bfseries HW2第4题pdf函数}
\titleformat*{\section}{\songti\zihao{4}\bfseries}
\titleformat*{\subsection}{\songti\zihao{5}\bfseries}
\renewcommand\thesection{\arabic{section}}
\author{王启骅 PB20020580}
\begin{document}
	\maketitle
	\section{题目}
	设pdf函数满足关系式
	\begin{center}
		$ p'(x)=a\delta(x)+b\ exp(-cx) ,x\in[-1,1],a\neq 0$
	\end{center}
	讨论该函数性质并给出抽样方法。
	\section{算法原理}
对p'(x)积分得到
\begin{equation}
	p(x)=a\theta(x)-\frac{b}{c}\ exp(-cx)+Const
\end{equation}


再次积分得到
	\begin{equation}
		\xi=\int_{-1}^{x}p(x)dx=
		\begin{cases}
			\frac{b}{c^2}[exp(-cx)-exp(c)]+Const(x+1)& \text{,$x<0 $}\\
			ax+\frac{b}{c^2}[exp(-cx)-exp(c)]+Const(x+1)& \text{,$x>0 $}
		\end{cases}
	\end{equation}
	
	
	将其归一化,得到
	\begin{equation}
		Const=\frac{1-\frac{b}{c^2}[e^{-c}-e^c]-a}{2}
	\end{equation}


	该方程为一个超越方程,没有解析解,之后可以通过二分法反解方程(2)得到$ x(\xi) $,其中$ \xi $对应均匀分布在[0,1]的随机数列,产生的数列x即为所求的抽样序列。
	
	
	或者通过舍选法,由于p(x)在[-1,1]上为有界函数,故一定有上界M,当a,b,c均>0时有$ M=a-\frac{b}{c}exp(-c)+Const $,并取
	\begin{equation}
		g(x,y)=\frac{1}{2M}
	\end{equation}
	\begin{equation}
		\xi_1=\frac{\xi_x+1}{2}
	\end{equation}
	\begin{equation}
		\xi_2=\frac{\xi_y}{M}
	\end{equation}
	其中$ \xi_1,\xi_2 $为在[0,1]上均匀分布的随机数序列,得到
	\begin{equation}
		\xi_x=2\xi_1-1
	\end{equation}
	\begin{equation}
		\xi_y=M\xi_2
	\end{equation}
	判断随机数列中的每一点
	\begin{equation}
		M\xi_2\le p(2\xi_1-1)
	\end{equation}
	若成立,则取$ x=2\xi_1-1 $


\end{document}